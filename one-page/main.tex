\documentclass[10pt, letterpaper]{article}

% Packages:
\usepackage[
    ignoreheadfoot, % set margins without considering header and footer
    top=2 cm, % seperation between body and page edge from the top
    bottom=2 cm, % seperation between body and page edge from the bottom
    left=2 cm, % seperation between body and page edge from the left
    right=2 cm, % seperation between body and page edge from the right
    footskip=1.0 cm, % seperation between body and footer
    % showframe % for debugging 
]{geometry} % for adjusting page geometry
\usepackage{titlesec} % for customizing section titles
\usepackage{tabularx} % for making tables with fixed width columns
\usepackage{array} % tabularx requires this
\usepackage[dvipsnames]{xcolor} % for coloring text
\definecolor{primaryColor}{RGB}{0, 0, 0} % define primary color
\usepackage{enumitem} % for customizing lists
\usepackage{fontawesome5} % for using icons
\usepackage{amsmath} % for math
\usepackage[
    pdftitle={John Doe's CV},
    pdfauthor={John Doe},
    pdfcreator={LaTeX with RenderCV},
    colorlinks=true,
    urlcolor=primaryColor
]{hyperref} % for links, metadata and bookmarks
\usepackage[pscoord]{eso-pic} % for floating text on the page
\usepackage{calc} % for calculating lengths
\usepackage{bookmark} % for bookmarks
\usepackage{lastpage} % for getting the total number of pages
\usepackage{changepage} % for one column entries (adjustwidth environment)
\usepackage{paracol} % for two and three column entries
\usepackage{ifthen} % for conditional statements
\usepackage{needspace} % for avoiding page brake right after the section title
\usepackage{iftex} % check if engine is pdflatex, xetex or luatex

% Ensure that generate pdf is machine readable/ATS parsable:
\ifPDFTeX
    \input{glyphtounicode}
    \pdfgentounicode=1
    \usepackage[T1]{fontenc}
    \usepackage[utf8]{inputenc}
    \usepackage{lmodern}
\fi

\usepackage{charter}

\usepackage{ragged2e}

% Some settings:
\raggedright
\AtBeginEnvironment{adjustwidth}{\partopsep0pt} % remove space before adjustwidth environment
\pagestyle{empty} % no header or footer
\setcounter{secnumdepth}{0} % no section numbering
\setlength{\parindent}{0pt} % no indentation
\setlength{\topskip}{0pt} % no top skip
\setlength{\columnsep}{0.15cm} % set column seperation
\pagenumbering{gobble} % no page numbering

\titleformat{\section}{\needspace{4\baselineskip}\bfseries\large}{}{0pt}{}[\vspace{1pt}\titlerule]

\titlespacing{\section}{
    % left space:
    -1pt
}{
    % top space:
    0.3 cm
}{
    % bottom space:
    0.2 cm
} % section title spacing

\renewcommand\labelitemi{$\vcenter{\hbox{\small$\bullet$}}$} % custom bullet points
\newenvironment{highlights}{
    \begin{itemize}[
        topsep=0.10 cm,
        parsep=0.10 cm,
        partopsep=0pt,
        itemsep=0pt,
        leftmargin=0 cm + 10pt     
    ]
}{
    \end{itemize}
} % new environment for highlights


\newenvironment{highlightsforbulletentries}{
    \begin{itemize}[
        topsep=0.10 cm,
        parsep=0.10 cm,
        partopsep=0pt,
        itemsep=0pt,
        leftmargin=10pt        
    ]
}{
    \end{itemize}
} % new environment for highlights for bullet entries

\newenvironment{onecolentry}{
    \begin{adjustwidth}{
        0 cm + 0.00001 cm
    }{
        0 cm + 0.00001 cm
    }
}{
    \end{adjustwidth}
} % new environment for one column entries

\newenvironment{twocolentry}[2][]{
    \onecolentry
    \def\secondColumn{#2}
    \setcolumnwidth{\fill, 4.5 cm}
    \begin{paracol}{2}
}{
    \switchcolumn \raggedleft \secondColumn
    \end{paracol}
    \endonecolentry
} % new environment for two column entries

\newenvironment{threecolentry}[3][]{
    \onecolentry
    \def\thirdColumn{#3}
    \setcolumnwidth{, \fill, 4.5 cm}
    \begin{paracol}{3}
    {\raggedright #2} \switchcolumn
}{
    \switchcolumn \raggedleft \thirdColumn
    \end{paracol}
    \endonecolentry
} % new environment for three column entries

\newenvironment{header}{
    \setlength{\topsep}{0pt}\par\kern\topsep\centering\linespread{1.5}
}{
    \par\kern\topsep
} % new environment for the header

\newcommand{\placelastupdatedtext}{% \placetextbox{<horizontal pos>}{<vertical pos>}{<stuff>}
  \AddToShipoutPictureFG*{% Add <stuff> to current page foreground
    \put(
        \LenToUnit{\paperwidth-2 cm-0 cm+0.05cm},
        \LenToUnit{\paperheight-1.0 cm}
    ){\vtop{{\null}\makebox[0pt][c]{
        \small\color{gray}\textit{Last updated in September 2024}\hspace{\widthof{Last updated in September 2024}}
    }}}%
  }%
}%

% save the original href command in a new command:
\let\hrefWithoutArrow\href

% new command for external links:


\begin{document}
    \newcommand{\AND}{\unskip
        \cleaders\copy\ANDbox\hskip\wd\ANDbox
        \ignorespaces
    }
    \newsavebox\ANDbox
    \sbox\ANDbox{$|$}

    \begin{header}
        \fontsize{25 pt}{25 pt}\selectfont Omkar Pradhan

        \vspace{5 pt}

        \normalsize
        \mbox{Alhambra, CA}%
        \kern 5.0 pt%
        \AND%
        \kern 5.0 pt%
        \mbox{\hrefWithoutArrow{mailto:omkar.pradhan@ieee.org}{omkar.pradhan@ieee.org}}%
        \kern 5.0 pt%
        \AND%
        \kern 5.0 pt%
        \mbox{\hrefWithoutArrow{tel:+1-215-407-5645}{1-215-407-5645}}%
        \kern 5.0 pt%
        \AND%
        \kern 5.0 pt%
        \mbox{\hrefWithoutArrow{https://linkedin.com/in/omkarpradhan}{linkedin.com/in/omkarpradhan}}%
    \end{header}

    \vspace{5 pt - 0.3 cm}


    \section{About Me}
    \justifying
        \begin{onecolentry}
            I have broad experience in RF system design, integration, and testing and am seeking an engineering leadership role. My preferred work location is within greater-LA area (on-site) or any US location (remote). I am a US person.
        \end{onecolentry}
    
    \section{Education}
        \begin{twocolentry}{
            \textbf{October 2019}
        }
            \textbf{University of Colorado}, PhD in Electrical Engineering\end{twocolentry}
        \vspace{0.10 cm}
        \begin{onecolentry}
            \begin{highlights}
                \item Thesis: Design of an endfire synthetic aperture radar for subsurface exploration of Europa, enceladus, and terrestrial glaciers (\href{https://https://www.proquest.com/openview/61a7e3296c08435c9df034c7a8b5c952/1?pq-origsite=gscholar&cbl=18750&diss=y}{link})
            \end{highlights}
        \end{onecolentry}
        \vspace{0.20cm}
        \begin{twocolentry}{
            \textbf{December 2013}
        }
            \textbf{University of Colorado}, MS in Electrical Engineering\end{twocolentry}
        \vspace{0.10 cm}
    
        \section{Experience}
        \begin{twocolentry}{
            \textbf{June 2021 – present}
        }
            \textbf{RF/Microwave Engineer}, \textit{NASA's Jet Propulsion Laboratory} -- Pasadena, CA
        \end{twocolentry}
        \vspace{0.30 cm}
        \begin{onecolentry}  
        \justifying
            \begin{highlights}
                \item Lead JPL's ASIC-based millimeter-wave radiometer technology development projects worth USD 2M, enabling 10x or more improvement in frequency bandwidth (>12 GHz), spectral resolution (<10 MHz), and SWaP metrics (<15W). This work contributed to USD 15M space-flight award under  \href{https://www.nasa.gov/news-release/nasa-selects-instruments-for-artemis-lunar-terrain-vehicle/}{NASA's Artemis program}.
                \item Rapidly developed, and deployed prototype airborne radiometers for NASA and NSF Earth science remote sensing campaigns, leading to USD 4M in directed funding from NASA for space-flight instrument.
                \item Served as a manager for NASA's Small Business Innovation Research (SBIR) program, representing JPL's strategic interests, and helping direct USD 500k worth of awards to private industry. 
                \item Mentored summer interns on radar and radiometer system testing and analyses.  
            \end{highlights}
        \end{onecolentry}

        \vspace{0.4 cm}

        \begin{twocolentry}{
            \textbf{October 2019 – May 2021}
        }
            \textbf{Postdoctoral Fellow}, \textit{NASA's Jet Propulsion Laboratory} -- Pasadena, CA
        \end{twocolentry}
        \vspace{0.30 cm}
        \begin{onecolentry}
            \begin{highlights}
                \item Designed a millimeter-wave differential absorption radar (DAR) for Martian remote sensing. This radar allows 10x improvement in near-surface vertical resolution of water vapor profiles. 
                \item Conceptualized and lead a 300k USD R\&D effort for detection of atmospheric turbulence using millimeter-wave transceivers, while mentoring doctoral student at Caltech. This experiment showcases for the first time RF-based turbulence remote sensing in a controlled laboratory environment at a fraction of the spatial scale.
            \end{highlights}
        \end{onecolentry}
    
    \section{Projects}
       \justifying
        \begin{twocolentry}{
            \textbf{April 2024 -- May 2025}
        }
            \textbf{Advanced Ultra-high Resolution Optical RAdiometer (AURORA)}\end{twocolentry}

        \vspace{0.20 cm}
        \begin{onecolentry}
            \begin{highlights}
                \item Instrument system engineer responsible for concept-through-implementation of a 110-190 GHz Earth sensing satellite-based radiometer, while leading a team of 5-6 peer engineers across institutions
            \end{highlights}
        \end{onecolentry}


        \vspace{0.3 cm}

        \begin{twocolentry}{
            \textbf{Feb 2022 -- present}
        }
            \textbf{NOAA-Advanced Millimeter-wave Sounder (NAMS)}\end{twocolentry}

        \vspace{0.20 cm}
        \begin{onecolentry}
            \begin{highlights}
                \item Lead design-through-implementation  of a millimeter-wave (60, 118, and 183 GHz) radiometer for NOAA's marine aviation operations
            \end{highlights}
        \end{onecolentry}


        \vspace{0.3 cm}

        \begin{twocolentry}{
            \textbf{Feb 2023 -- May 2024}
        }
            \textbf{Microwave Electrojet Magnetogram (MEM)}\end{twocolentry}

        \vspace{0.20 cm}
        \begin{onecolentry}
            \begin{highlights}
                \item Developed automated pre-launch polarimetric calibration procedure for JPL's 118 GHz polarimetric radiometer currently in low Earth orbit as part of NASA's \href{https://science.nasa.gov/mission/ezie/}{EZIE} mission                
            \end{highlights}
        \end{onecolentry}

    \section{Tools and Technologies}        
        \begin{onecolentry}
            \textbf{Languages/Scripting:} C, Tool Command Language (TCL), Linux Shell, Python, Matlab
        \end{onecolentry}
        \vspace{0.2 cm}
        \begin{onecolentry}
            \textbf{Software:} Familiar with Keysight ADS, Ansoft HFSS, TICRA Grasp, Altium, Solid Works
        \end{onecolentry}
        \vspace{0.2 cm}
        \begin{onecolentry}
            \textbf{Hardware:}  Experienced in using Vector Network Analyzers, Phase Analyzers, Spectrum Analyzers, Power Meters
        \end{onecolentry} 

    
       


    % \section{Publications}
        
    %     \begin{samepage}
    %         \begin{twocolentry}{
    %             Jan 2004
    %         }
    %             \textbf{3D Finite Element Analysis of No-Insulation Coils}
    %         \end{twocolentry}

    %         \vspace{0.10 cm}
            
    %         \begin{onecolentry}
    %             \mbox{Frodo Baggins}, \mbox{\textbf{\textit{John Doe}}}, \mbox{Samwise Gamgee}

    %             \vspace{0.10 cm}
                
    %     \href{https://doi.org/10.1109/TASC.2023.3340648}{10.1109/TASC.2023.3340648}
    %     \end{onecolentry}
    %     \end{samepage}
    

\end{document}